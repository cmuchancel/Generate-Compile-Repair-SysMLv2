% !TEX encoding = UTF-8 Unicode
%%%%%%%%%%%%%%%%%%%%%%%%%%%%%%%%%%%%%%%%%%%%%%%%%%%%%%%%%%%%%%%%%%%%%%%%%%%%%%%%%%%%%%%%%%%%%%%%%%%%%%% 
%%
%%  This file is asmeconf-template.tex, a LaTeX template to format ASME Conference papers according to
%%  the requirements on ASME's conference web pagess, and including hypertext support for the pdf.
%%
%%  This file is version 1.45 dated 2025/11/10
%%  
%%  As of version 1.11, this template defaults to ASME's newer conference guidelines first posted July 2019.
%% 			Those guidelines changed the author block formatting to be inline. 
%%			If you prefer the traditional grid format, use the class option [grid].
%%			Nomenclature now follows the abstract, and the abstract text is set in italics.
%%
%%  Author: John H. Lienhard, V
%%          Department of Mechanical Engineering
%%          Massachusetts Institute of Technology
%%          Cambridge, MA 02139-4307 USA
%%
%%  Class options include:
%%
%%          * An option to use the traditional grid arrangement of author names, [grid].  
%%			*	 With this option, line breaks (\\) may be inserted into the address as needed. 
%%			*	 Author names that include commas should be enclosed in braces, e.g.,  {Joseph L. Smith, Jr.}.  
%%			*	 Authors may be grouped above a single affiliation using braces, e.g., {Henry Tudor, Catherine Parr}.
%%
%%          * An option to balance the heights of columns on the last page, [balance]. 
%%
%%          * An option to include line numbers, [lineno]. You must *run twice* for proper placement of the line numbers. 
%%			*	 The lineno package does not number titles, footnotes, captions, or tables.
%%          *    This option will disable balancing column height on final page if that option has been invoked.
%%          *    The lineno package won't always number the lines preceding displayed math in a paragraph because
%%          *    paragraph has not ended.  See that package's documentation for macros to address this problem, or
%%          *    just leave a blank line above the displayed equation while you are editing and then remove the 
%%          *    blank line along with [lineno] option when you move to your final version.
%%
%%			* An option not to use a boldface font for caption text, [unboldcaption]
%%
%%          * Options to: 	omit the ASME copyright footer, [nofoot]; 
%%			*				use government employee copyright notice,      [govt];
%%			*				use government contractor copyright notice,    [contractor];
%%			*				use copyright notice for some gov't employees, [somegovt];
%%			*				omit the conference headers on the titlepage,  [nohead]
%%
%%			* Option to use LuaLaTeX **without** unicode-math and fontspec, [nofontspec]
%%
%%          * Math options from M. Sharpe's newtxmath package: upright integrals [upint]; 
%%          *    [varvw] for a v and w that are better distinguished from Greek nu; and also 
%%          *    [subscriptcorrection, smallerops, varg, frenchmath, varbb, cmbraces, slantedGreek,...] 
%%			* 	 See newtx documentation for descriptions (at CTAN: https://www.ctan.org/pkg/newtx, v1.744 or higher is best).
%%			*	 newtx is not loaded with lualatex unless the [nofontspec] option is used.
%%
%%          * An option to allow hyphenation of the typewriter font [hyphenate], from the inconsolata package (pdfTeX only).
%%          *    Hyphenation is normally suppressed for a typewriter (monospaced) font because those are often used for code.
%%			*	 To replace the default variable word spacing by monospacing, use the option [mono].
%%			*	 To get a zero without a slash, use [var0]
%%
%%			* PDF/A compliance:  Since 2022, LaTeX has included support for PDF/A, through the \DocumentMetadata{..} command.
%%
%%          * Options (used by the babel package) to include passages in languages other than English (e.g., a translation 
%%			*    of the abstract).  See Appendix C for details.
%%			*    	pdftex: Appropriate scripts will be loaded if you call the class options [greek], [russian], or [vietnamese].
%%			*    
%%			*    	lualatex: Language support is most extensive when running LuaLaTeX, which automatically loads fontspec.
%%			*		Non-Latin scripts require the [loadscripts] option, see Appendix C. 
%%			*	 	Be aware that you might need to install additional fonts for some languages.
%%			 
%%  The use of commands defined or modified by the asmeconf class is illustrated throughout this file. In particular, 
%%  ASME requires capitalized section headings, and as a result some care is needed when using macros in section headings,
%%  as also illustrated below.
%%
%%  Use an **up-to-date and complete** installation of LaTeX, such as TeX Live 2025 or later. 
%%		LaTeX formats earlier than 2022 are not fully supported.
%%
 %=========================================================
%% 
%% LICENSE: 
%%
%% Copyright (c) 2025 John H. Lienhard
%%
%% Offered under the MIT license: https://ctan.org/license/mit 
%%
%%%%%%%%%%%%%%%%%%%%%%%%%%%%%%%%%%%%%%%%%%%%%%%%%%%%%%%%%%%%%%%%%%%%%%%%%%%%%%%%%%%%%%%%%%%%%%%%%%%%%%% 

%%  RECOMMENDED pdf management code.
%%  This code was added to the LaTeX kernel in June 2022.
%% 		See https://www.latex-project.org/news/latex2e-news/ltnews35.pdf
%%  If you have problems with these lines, you can comment them out or, better, update your LaTeX format.

\DocumentMetadata{
	lang		= eng-US, 	% default
	pdfstandard	= A-3u,		% A-2b, A-2u, A-3b, A-3u. Check that your figures also meet the standard you choose.
	pdfversion	= 1.7, 		% an older version of PDF, but very widely supported
%	pdfversion  = 2.0, 		% default for LaTeX. Version 2.0 supports tagged PDF.
%	pdfstandard = { ua-2 , a-4f }, % run with lualatex and use a very recent LaTeX format (e.g., 2025/11/01)
%	tagging 	= on, 		% for producing tagged pdf
}

%%%%%%%%%%%%%%%%%%%%%%%%%%%%%%%%%%%%%%%%%%%%%%%%%%%%%%%%%%%%%%%%%%%%%%%%%%%%%%%%%%%%%%%%%%%%%%%%%%%%%%%

%% CLASS OPTIONS are described above. Change the options given below to meet your needs.
%% 
%%	 	NB: Remove the [colorlinks] option before *final* submission to ASME, to get black text for printing,
%%			but keep that option for electronic use.
%%
%%		NB: If you are not using the language options, * remove * them (together with Appendices C and D).
%%			Greek, cyrillic languages, and vietnamese if used, must be named as \documentclass options under pdftex.
%%			Spanish, german, and many others, if used, do not need to be named when the babel package is dated 2024 or later.
%%
%%		NB: the mathalfa option was dropped in v1.41; load the mathalpha package in your preamble instead.

\documentclass[colorlinks,upint,subscriptcorrection,varvw,hyphenate,balance]{asmeconf} % pdftex
\usepackage{tikz}
\usetikzlibrary{arrows.meta,positioning,calc,shapes.geometric}
\usepackage{placeins}
\usepackage{float}

%\documentclass[colorlinks,upint,subscriptcorrection,varvw,balance,loadscripts,german,vietamese]{asmeconf} % lualatex

%%%%%  pdf metadata  %%%%%%%%%%%%%%%%%%%%%%%%%%%%%%%%%%%%%%%%%%%%%%%%%%%%%%%%%%%%%%%%%%%%%%%%%%%%%%%%%%

\hypersetup{%
	pdfauthor={John H. Lienhard},									  % <=== change to YOUR name
	pdftitle={ASME Conference Paper LaTeX Template},                  % <=== change to YOUR pdf file title
	pdfkeywords={ASME conference paper, LaTeX template, BibTeX style},% <=== change to YOUR pdf keywords
	pdfsubject = {Describes the asmeconf LaTeX template},			  % <=== change to YOUR subject
%	pdfurl={https://ctan.org/pkg/asmeconf},% may delete
%	pdflicenseurl={https://ctan.org/pkg/asmeconf},% may delete
}
% If an author name or the title include a comma, enclosed it in braces, e.g., pdfauthor{John Forbes Nash{,} Jr.}


%%%%%%%%%%%%%%%%%%%%%%%%%%%%%%%%%%%%%%%%%%%%%%%%%%%%%%%%%%%%%%%%%%%%%%%%%%%%%%%%%%%%%%%%%%%%%%%%%%%%%%%

\allowdisplaybreaks % from amsmath package, allows multiline equations to break across pages (delete if not wanted)
					% using \\* instead of \\ will prevent specific lines from being pagebreaks.

% logos used in this document
\newcommand*\pdfTeX{pdf\TeX}    
\newcommand*\LuaLaTeX{Lua\LaTeX}
\newcommand*\BibTeX{Bib\TeX}

\begin{document}

% Change these fields to the right content for your conference.
% You can comment these out if for some reason you don't want a header.
% Use title case for the conference name (first letters capitalized), not all capitals

\ConfName{Proceedings of the ASME 2026\linebreak International Design Engineering Technical Conferences \&\linebreak Computers and Information in Engineering Conference}
\ConfAcronym{IDETC-CIE 2026}
\ConfDate{August 23--26, 2026}
\ConfCity{Houston, TX, USA}
\PaperNo{DETC2026-XXXX}

% Units of measure (e.g., cm) and other specialty lowercase terms in the title should be 
%   enclosed in \NoCaseChange{...} to maintain lower case type.
%   The rest of the title will automatically be set in all capital letters.
%
%	\title{Place Title Here: Place Subtitle After Colon} 

\title{Reliable Natural-Language to SysMLv2 Translation via Compiler-Driven Iterative Refinement} % <=== replace with YOUR title
 
%   Put author names into the order you want. Use the same order for affiliations.
%   \affil{#} tags the author's affiliation to the address in \SetAffiliation{#}.
%   No space between last name and \affil{#}, separate names with commas.
%
%	For a sole author or a single affiliation for all authors, {#} may be left empty, i.e. \affil{} and \SetAffiliation{} (but not with [grid] option!)
%
%   \CorrespondingAuthor{email} follows that author's affiliation, no spaces before or after 
%   If multiple corresponding authors, put both email addresses in the same command and place after both authors.
%
%   \JointFirstAuthor, if applicable, follows the affiliation of the relevant authors, no spaces.

\SetAuthors{%
    Chance LaVoie\affil{1}\CorrespondingAuthor{chancel@cmu.edu},
    Eladio Andujar Lugo\affil{1},
    Levent Burak Kara\affil{1}
}

\SetAffiliation{1}{Department of Mechanical Engineering, Carnegie Mellon University, Pittsburgh, PA 15213 USA}
%   Note: You can force a line break in the address using \\ 

%	To switch from inline author names to gridded names, use the [grid] option.

\maketitle

%%% Use this footnote for tracking various versions of your draft. Change text to suit your own needs. 
%%% \date{..} calls the same command. 
\versionfootnote{Documentation for \texttt{asmeconf.cls}: Version~\versionno, \today.}% <=== Delete before final submission.

%%% Change the following to your keywords.  Keywords are automatically printed at the end of the abstract.
%%% This command MUST COME BEFORE the end of the abstract.
%%% If you don't want keywords, leave the argument of \keywords{} empty (or use the abstract* environment)

\keywords{SysMLv2, model-based systems engineering, large language models, compiler-in-the-loop, natural language to model translation, syntactic validity}

%%%%%  End of fields to be completed. Now write your paper. %%%%%%%%%%%%%%%%%%%%%%%%%%%%%%%%%%%%%%%%%%%


%%%%%  ABSTRACT  %%%%%%%%%%%%%%%%%%%%%%%%%%%%%%%%%%%%%%%%%%%%%%%%%%%
%%
%% Abstract should be 200 words or less
\begin{abstract}
SysMLv2 is a textual, compilable modeling language, so generated output is only useful if it passes deterministic compiler checks. One-shot large language model generation frequently produces plausible but uncompilable SysMLv2, motivating compiler-gated refinement rather than one-pass synthesis.

This paper studies a syntax-only compiler-in-the-loop pipeline on the full SysMBench prompt set (IDs 1--151), evaluated across four model configurations. For each prompt/model trial, we compare baseline single-shot output (iteration 1 only) against the final output after iterative generate--compile--repair using \texttt{syside check} as the oracle. Across 604 trials, first-shot compilation succeeds on 51.16\% (309/604; 95\% Wilson CI: 47.18--55.13\%), while the compiler-in-the-loop final output reaches 100.00\% eventual compilation (604/604; 95\% Wilson CI: 99.37--100.00\%), a gain of 48.84 percentage points with zero unresolved prompts. Failed-first-shot cases recover fully (295/295). These results show that deterministic compiler feedback is the primary control signal for reliable natural-language--to--SysMLv2 generation and that single-shot generation leaves a large, recoverable syntactic error surface.
\end{abstract}

%%%%%%%%%  BODY OF PAPER %%%%%%%%%%%%%%%%%%%%%%%%%%%%%%%%%

\section{Introduction}
Model-Based Systems Engineering (MBSE) relies on formal, machine-checkable artifacts to support early design iteration, requirements traceability, and verification. The release of Systems Modeling Language version~2 (SysMLv2)~\cite{omgSysMLv2Spec2024} strengthens this workflow...
  strengthens this workflow by introducing a fully textual modeling language with a well-defined grammar and an executable toolchain, enabling models to be parsed, compiled, and version-controlled like software.

This shift creates an opportunity for automated model synthesis: engineers could describe a system in natural language and obtain a SysMLv2 model that can be compiled and analyzed. Large language models (LLMs) are a natural candidate for this translation task, given their success at generating structured code from natural-language specifications. However, a model that is ``almost correct'' is not useful in practice: even small syntax or metamodel errors prevent compilation and therefore block rendering, visualization, and downstream analysis. Reliable translation therefore requires \emph{syntactic validity}, not just plausible-looking output.

The current status quo is that LLMs can often produce readable pseudo-code or SysML-like text. Pseudo-code is not sufficient for MBSE: SysML is intended to serve as a \emph{compilable} and \emph{visualizable} system representation, and compilation is the gateway to rendering, visualization, and analysis inside engineering toolchains. If an LLM output is not syntactically correct, it cannot be compiled and therefore cannot be rendered into the diagrams and model views that engineers use to understand and communicate system structure and behavior. As a result, the practical value of AI-assisted modeling hinges on producing syntactically valid SysMLv2, not informal approximations.

In many code-generation domains, syntactic reliability can be improved by training or fine-tuning on large corpora of valid programs. For SysMLv2, such data is currently scarce: there is no large, broadly available dataset of compiler-verified models, and the language itself is evolving. Consequently, one-shot LLM generations often mix SysML~v1 patterns with SysMLv2 constructs or violate the SysMLv2 grammar, leading to outputs that read convincingly but fail deterministic checks. This data scarcity motivates a method that improves correctness \emph{without} modifying model weights.

This paper presents a compiler-in-the-loop framework that couples an off-the-shelf LLM with a deterministic verifier in an iterative generate--compile--repair loop. At each iteration, the model proposes a complete SysMLv2 candidate; the verifier returns precise diagnostics (error types and locations); and a controller converts these diagnostics into targeted revision instructions. The loop terminates only when zero errors are reported, yielding syntactic validity by construction without training, fine-tuning, or specialized hardware.

We evaluate the approach on all 151 SysMBench prompts and compare two conditions for each prompt/model pair: single-shot compilation at iteration 1 versus final compilation after compiler-in-the-loop refinement. Results show that deterministic compiler feedback is the primary driver of dependable generation: iterative refinement converts many first-pass failures into compilable SysMLv2 while single-shot generation alone leaves a large recoverable error surface.

\section{Related Work}

Our review focuses on prior efforts in automated SysML v2 generation, grammar-constrained structured synthesis, and compiler-guided iterative refinement, with an emphasis on approaches addressing reliable syntactic correctness in low-data modeling languages.

\subsection{LLM-Based SysML v2 Model Generation}
Recent work has explored using large language models to generate SysML v2 models from natural language. Bouamra et al.~\cite{bouamra2025systemp} propose SysTemp, a multi-agent framework that decomposes the generation task into requirement extraction, template-based skeleton construction, and grammar-level parsing feedback. Their system employs a ParserAgent to detect syntactic errors and iteratively repair generated models, achieving 80\% convergence across five evaluation scenarios. This work demonstrates that structural scaffolding and agent decomposition improve grammar conformity in sparse-data settings. However, syntactic correctness is defined at the grammar level and convergence is reported empirically rather than enforced as a termination invariant.

Cibrián et al.~\cite{cibrian2025sysmlagent} introduce an agent-based framework combining retrieval-augmented generation (RAG) with ANTLR-based grammar validation for SysML v2 synthesis. Their approach reports 100\% syntactic validity across 20 curated prompts under grammar-level parsing. While this represents a significant advance in structured generation reliability, correctness is enforced through grammar parsing rather than full compiler validation. Because grammar-level validation ensures context-free structural conformity but does not enforce full static semantics under the official toolchain, grammar-valid models may still fail compilation in production environments.
neccessary but not sufficient
Together, these efforts establish that iterative validation substantially improves syntactic success in SysML v2 generation. However, prior systems define correctness at the grammar level and evaluate on limited scenario sets. Our work builds on these insights while shifting the correctness oracle from grammar parsing to the official SysIDE compiler and scaling evaluation to benchmark-level scenarios.

\subsection{Grammar-Constrained and Template-Based Structured Synthesis}

Structured synthesis approaches aim to reduce hallucinations in LLM output by constraining generation via templates or grammar rules. SysTemp~\cite{bouamra2025systemp} explicitly uses a template generator based on Jinja2 to construct syntactically compliant model skeletons prior to completion. Grammar-constrained decoding and post-generation parsing similarly reduce token-level structural invalidity.

While grammar validation ensures adherence to context-free rules, it does not enforce type resolution, cross-reference integrity, constraint satisfaction, or toolchain compatibility. Thus, grammar-valid artifacts may remain unusable within industrial MBSE workflows. Our approach differs by treating grammar conformity as necessary but insufficient and requiring full compiler acceptance prior to termination.

\subsection{Compiler-in-the-Loop and Verifier-Guided Generation}

Prior work in neural code generation has explored leveraging compiler diagnostics to improve the compilability of model outputs. For example, Wang et al.~\cite{wang2022compilable} propose a multi-stage refinement framework that uses compiler feedback to iteratively revise generated programs and increase compilation success rates. Such approaches demonstrate that deterministic compiler signals can serve as effective supervisory feedback in programming-language settings. However, these systems operate in mature programming ecosystems and treat compilation success as an empirical objective rather than as a structural termination invariant.

Beyond compiler-feedback approaches in code generation, iterative generation guided by deterministic verifiers has also been studied under counterexample-guided inductive synthesis and execution-based refinement paradigms. Grubišić et al.~\cite{grubisic2024compiler} demonstrate that LLVM compiler feedback can serve as a supervisory signal for refining LLM-generated intermediate representation. Their work shows that deterministic compilation signals constrain output space and improve structural validity in programming languages.

While these approaches establish the feasibility of compiler-aware refinement in software domains, they differ in both objective and context from model-based systems engineering. Compiler feedback in such systems is typically used to improve optimization quality or increase compilation probability, rather than to enforce strict termination conditions. Moreover, these methods operate in programming languages with extensive training corpora and stable ecosystems.

In contrast, our work applies compiler-in-the-loop refinement to SysMLv2, a newly standardized modeling language with sparse representation in LLM training data and strict toolchain requirements. Rather than treating compilation results as heuristic feedback, we enforce zero-error compilation under the SysMLv2 compiler (invoked via \texttt{syside check}) as a termination invariant. This elevates syntactic correctness from an empirical metric to a property guaranteed by construction, aligning generation reliability with production MBSE toolchain criteria rather than grammar-level approximations.


\section{Methodology}

\subsection{Study Objective and Paired Design}
This study evaluates one question: for the same prompt and the same model, does validator-in-the-loop refinement increase production validation acceptance relative to single-shot generation? The scope is strictly syntactic.

The experimental unit is one prompt--model pair. For each unit, we evaluate two paired conditions taken from the same run trajectory:
\begin{enumerate}
    \item \textbf{Baseline (single-shot):} production validation outcome at iteration 1 only.
    \item \textbf{Pipeline (iterative):} production validation outcome at the final available iteration after iterative repair.
\end{enumerate}

Because both outcomes are taken from the same prompt--model run, this design isolates the effect of iterative validator feedback while holding prompt content and model identity fixed.

\subsection{Validator-in-the-Loop Generation Procedure}
Our controller follows a generate--validate--repair workflow for natural-language--to--SysMLv2 generation. At each iteration, the model proposes a complete SysMLv2 candidate, the production validator returns deterministic diagnostics, and the next model call is conditioned on those diagnostics.

Let $P$ denote the natural-language prompt, $M_t$ the generated candidate at iteration $t$, and $V(\cdot)$ the production validator. The update is
\[
M_{t+1} = f\!\left(P, M_t, V(M_t)\right),
\]
where $f(\cdot)$ is the model revision operator conditioned on validator feedback.

\begin{figure}[H]
\centering
\resizebox{\columnwidth}{!}{\input{gcr_loop}}
\caption{Validator-in-the-loop generate--validate--repair workflow. The prompt is fixed per case, and each revision is driven by deterministic validator diagnostics.}
\label{fig:gcr_loop_methods}
\end{figure}

The production validator oracle is SysIDE validation (\texttt{syside check})~\cite{sensmetry2024syside}. A run is successful and terminates only when zero validation errors are reported.
This oracle choice is additionally supported by an auxiliary ten-case demonstration in the repository showing ANTLR parser pass with production-validation failure, i.e., grammar conformity without operational acceptability~\cite{antlrVsSysideDemo2026}.

\subsection{Dataset, Model Coverage, and Outcome Extraction}
SysMBench provides paired natural-language prompts and ground-truth SysMLv2 models for benchmark evaluation~\cite{jin2025sysmbench}. In this study, we use the curated natural-language prompt set (IDs 1--151) as generation inputs because it was designed to stress SysMLv2 LLM generation across diverse modeling patterns.

To assess model-agnostic behavior of the same controller, we run four model configurations: OpenAI Codex 5.2 (\texttt{gpt-5.2-codex})~\cite{openaiGPT52Codex2026}, Anthropic Sonnet 4.6 (\texttt{claude-sonnet-4-6})~\cite{anthropicSonnet46API2026}, DeepSeek Reasoner (\texttt{deepseek-reasoner})~\cite{deepseekReasonerAPI2026}, and Mistral Large (\texttt{mistral-large-latest})~\cite{mistralLargeLatestDocs2026}. This yields 604 prompt-level cases (151 prompts $\times$ 4 models).

From saved run records, we extract single-shot and pipeline pass/fail outcomes, iterations run, iterations to success, first/final error counts, cumulative error counts, per-iteration runtime, and token usage. Error families are grouped from validator diagnostics (for example, parsing and reference errors), while warnings are tracked separately and do not change pass/fail labels.

\subsection{Endpoints}
The primary endpoint is production validation acceptance. For each analysis slice (overall and per-model), we report:
\begin{enumerate}
    \item single-shot pass rate (iteration 1 only),
    \item pipeline pass rate (final iteration of the validator-gated loop).
\end{enumerate}

Secondary diagnostics include iterations-to-success, first-iteration and cumulative error counts, and validator error-family frequencies. These secondary diagnostics are used for characterization, not for primary efficacy claims.

\subsection{Convergence Metrics and Rate Characterization}
To characterize how quickly the loop approaches acceptance, we index progress by repair cycles. Let $k=0$ denote the initial single-shot generation (no validator feedback), and let $k\geq 1$ denote $k$ rounds of generate--validate--repair. Let $A_k$ denote cumulative production-validation acceptance after repair cycle $k$.

We define residual failure mass as
\[
R_k = 1 - A_k,
\]
where $R_k$ is the fraction of prompt--model cases not yet accepted at repair cycle $k$. We then define an empirical contraction ratio
\[
\rho_k = \frac{R_{k+1}}{R_k}.
\]
When $0 < \rho_k < 1$, residual failure shrinks from one cycle to the next; when $\rho_k$ is approximately stable across $k$, the observed trajectory suggests contraction-like behavior, with approximately multiplicative reduction in the early repair regime under standard contraction-style analyses in iterative methods~\cite{banach1922operations,varga2009matrix,nocedal2006numerical,he2016averageConvergenceRate}.
In finite empirical campaigns, we estimate contraction behavior from early-to-mid cycles, before residual mass becomes very small; terminal transitions are excluded from rate estimation because finite-sample tail effects can produce unstable ratios near the finite-sample resolution.
This characterization is descriptive of observed finite-sample behavior and does not assert formal convergence guarantees.

We also report time-to-threshold acceptance
\[
T_{\varepsilon} = \min\{k : A_k \ge 1-\varepsilon\},
\]
with explicit reporting of $T_{90}$, $T_{95}$, and $T_{99}$. This follows standard iteration-to-threshold (iteration-to-$\varepsilon$) complexity summaries in optimization~\cite{polyak1987introductionOptimization}. These metrics provide an interpretable rate summary complementary to endpoint pass rates.

This framing follows empirical convergence-rate characterizations used in iterative optimization and search, including multiplicative (geometric-style) interpretations of residual reduction~\cite{he2016averageConvergenceRate}, while remaining descriptive rather than proving formal convergence guarantees. It is also consistent with iterative verifier-feedback refinement in code generation, where deterministic diagnostics guide successive corrections toward acceptance~\cite{wang2022compilable,grubisic2024compiler}.

\subsection{Statistical Analysis}
For convergence reliability, we model each prompt--model case as a Bernoulli success if it reaches eventual production-validator acceptance (zero SysIDE errors) within the allowed loop. Let $p$ denote convergence probability for prompts drawn from the SysMBench-style benchmark distribution under the same controller, validator, and backend configuration. Because the validator-gated loop is monotonic---accepted cases terminate immediately---regression transitions (single-shot pass $\rightarrow$ pipeline fail) are structurally excluded. Improvement is therefore quantified descriptively rather than via symmetric paired hypothesis tests.

We report an exact 95\% Clopper--Pearson lower confidence bound for $p$~\cite{clopper1934confidenceLimits}; in the all-success case this is $p_L=(\alpha/2)^{1/n}$ with $\alpha=0.05$. We also report the complementary one-sided 95\% binomial upper confidence bound on failure probability $q=1-p$. In the zero-failure case, this bound is obtained by solving $(1-q)^n=0.05$, yielding $q_U=1-0.05^{1/n}$. For large $n$, this expression is well approximated by $q_U\approx-\ln(0.05)/n\approx 3/n$.

These reliability bounds are intentionally scoped to SysMBench-style prompt distributions under the evaluated configuration and are reported numerically in Results. They are not treated as universal guarantees over arbitrary natural-language inputs.

\subsection{Reproducibility}
All code, run artifacts, and analysis outputs used in this study are stored in the project repository~\cite{sysmbenchCompilerLoopRepo2026}. The repository includes the scripts required to regenerate campaign statistics, tables, and figures.


\section{Results}

\subsection{Primary Outcome: Production Validation Acceptance}
Across all 604 prompt-level trials (151 prompts for each of four models), single-shot production validation succeeded for 309/604 cases (51.16\%). Under validator-in-the-loop refinement, pipeline production validation succeeded for 604/604 cases (100.00\%).

\begin{figure}[H]
\centering
% Figure 2 options (pick one):
% \definecolor{singleShotColor}{RGB}{196,106,28}
\definecolor{pipelineColor}{RGB}{44,138,100}
\begin{tikzpicture}
\begin{axis}[
    width=\columnwidth,
    height=0.34\columnwidth,
    xmin=0,
    xmax=105,
    ymin=0.4,
    ymax=1.6,
    xlabel={Acceptance Rate (\%)},
    ytick={1},
    yticklabels={Overall},
    axis lines*=left,
    xmajorgrids=true,
    grid style={dashed,gray!25},
    tick label style={font=\footnotesize},
    label style={font=\footnotesize},
    legend style={
        draw=none,
        font=\footnotesize,
        at={(0.5,1.03)},
        anchor=south,
        legend columns=2
    }
]
\addplot+[very thick,gray!55,mark=none,forget plot] coordinates {(51.16,1) (100.00,1)};

\addplot+[
    only marks,
    mark=square*,
    mark size=2.8pt,
    fill=singleShotColor,
    draw=singleShotColor!70!black,
    nodes near coords,
    nodes near coords style={font=\footnotesize, text=black, anchor=south, yshift=1pt},
] coordinates {(51.16,1)};

\addplot+[
    only marks,
    mark=square*,
    mark size=2.8pt,
    fill=pipelineColor,
    draw=pipelineColor!70!black,
    nodes near coords,
    nodes near coords style={font=\footnotesize, text=black, anchor=south, yshift=1pt},
] coordinates {(100.00,1)};

\legend{Single-shot,Pipeline}
\end{axis}
\end{tikzpicture}

% \definecolor{singleShotColor}{RGB}{196,106,28}
\definecolor{pipelineColor}{RGB}{44,138,100}
\begin{tikzpicture}
\begin{axis}[
    ybar stacked,
    bar width=24pt,
    width=\columnwidth,
    height=0.52\columnwidth,
    ymin=0,
    ymax=105,
    ylabel={Acceptance Rate (\%)},
    symbolic x coords={Overall},
    xtick=data,
    xticklabels={Overall (single-shot to pipeline)},
    x tick label style={font=\footnotesize, align=center},
    axis lines*=left,
    ymajorgrids=true,
    grid style={dashed,gray!25},
    tick label style={font=\footnotesize},
    label style={font=\footnotesize},
    legend style={
        draw=none,
        font=\footnotesize,
        at={(0.5,1.02)},
        anchor=south,
        legend columns=2
    },
]
\addplot+[
    fill=singleShotColor,
    draw=singleShotColor!70!black,
    nodes near coords,
    every node near coord/.append style={font=\footnotesize, text=black, anchor=south, yshift=1pt},
] coordinates {(Overall,51.16)};

\addplot+[
    fill=pipelineColor,
    draw=pipelineColor!70!black,
    nodes near coords,
    every node near coord/.append style={font=\footnotesize, text=black, anchor=south, yshift=1pt},
    point meta=explicit symbolic,
    nodes near coords={+\pgfmathprintnumber[fixed,precision=2]{\pgfplotspointmeta}},
] coordinates {(Overall,48.84) [48.84]};

% Top label (pipeline total)
\node[font=\footnotesize] at (axis cs:Overall,102.0) {100.00};

\legend{Single-shot accepted,Recovered by pipeline}
\end{axis}
\end{tikzpicture}

\definecolor{singleShotColor}{RGB}{196,106,28}
\definecolor{pipelineColor}{RGB}{44,138,100}
\begin{tikzpicture}
\begin{axis}[
    xbar,
    width=0.97\columnwidth,
    height=0.30\columnwidth,
    xmin=0,
    xmax=105,
    xlabel={Acceptance Rate (\%)},
    symbolic y coords={Pipeline,Single-shot},
    ytick={Single-shot,Pipeline},
    yticklabels={Single-shot,Pipeline},
    enlarge y limits={abs=4pt},
    axis lines*=left,
    xmajorgrids=true,
    grid style={dashed,gray!25},
    tick label style={font=\footnotesize},
    label style={font=\footnotesize},
    nodes near coords,
    every node near coord/.append style={font=\footnotesize, text=black, anchor=west, xshift=2pt},
]
\addplot+[fill=singleShotColor, draw=singleShotColor!70!black] coordinates {
    (51.16,Single-shot)
};
\addplot+[fill=pipelineColor, draw=pipelineColor!70!black] coordinates {
    (100.00,Pipeline)
};
\end{axis}
\end{tikzpicture}

\caption{Overall single-shot vs final pipeline production-validation acceptance across all 604 prompt-level cases.}
\label{fig:overall_baseline_vs_pipeline}
\end{figure}

\subsection{Per-Model Reliability}
All models reached 151/151 eventual production validation under the validator-gated loop, but single-shot pass rates differed substantially. Figure~\ref{fig:by_model_baseline_vs_pipeline} provides the per-model comparison with exact acceptance values annotated on the bars.

\begin{figure}[H]
\centering
\definecolor{singleShotColor}{RGB}{196,106,28}
\definecolor{pipelineColor}{RGB}{44,138,100}
\begin{tikzpicture}
\begin{axis}[
    ybar,
    bar width=13pt,
    width=\columnwidth,
    height=0.60\columnwidth,
    ymin=0,
    ymax=110,
    ylabel={Acceptance Rate (\%)},
    symbolic x coords={Sonnet,OpenAI,DeepSeek,Mistral},
    xtick=data,
    xticklabel style={font=\footnotesize, rotate=0, anchor=north},
    nodes near coords,
    nodes near coords={\pgfmathprintnumber[fixed,precision=2]{\pgfplotspointmeta}},
    nodes near coords align={center},
    nodes near coords style={
        font=\tiny,
        text=black,
    },
    every node near coord/.append style={anchor=south, yshift=-1.8pt},
    ymajorgrids=false,
    enlarge x limits=0.16,
    axis lines*=left,
    tick label style={font=\footnotesize},
    label style={font=\footnotesize},
    legend style={
        draw=none,
        font=\footnotesize,
        at={(0.5,1.04)},
        anchor=south,
        legend columns=2
    },
    legend cell align={left},
    legend image code/.code={
        \draw[#1] (0cm,-0.075cm) rectangle (0.22cm,0.075cm);
    },
]
\addplot+[fill=singleShotColor, draw=singleShotColor!70!black] coordinates {
    (Sonnet,82.78)
    (OpenAI,41.72)
    (DeepSeek,41.06)
    (Mistral,39.07)
};
\addplot+[fill=pipelineColor, draw=pipelineColor!70!black] coordinates {
    (Sonnet,100.00)
    (OpenAI,100.00)
    (DeepSeek,100.00)
    (Mistral,100.00)
};
\legend{Single-shot (iter 1),Pipeline (final)}
\end{axis}
\end{tikzpicture}

\caption{Per-model single-shot vs final pipeline production-validation acceptance.}
\label{fig:by_model_baseline_vs_pipeline}
\end{figure}

Anthropic Sonnet 4.6 had the highest single-shot pass rate (82.78\%), while OpenAI (41.72\%), DeepSeek Reasoner (41.06\%), and Mistral Large (39.07\%) showed worse single-shot behavior and larger single-shot-to-pipeline gaps.

\subsection{Statistical Significance}
For overall proportions, single-shot pass rate was 51.16\% (95\% Wilson CI: 47.18--55.13\%), and pipeline pass rate was 100.00\% (95\% Wilson CI: 99.37--100.00\%).

Because baseline and pipeline outcomes are paired on the same prompt--model cases, we used McNemar's test. The observed discordant counts were $b=0$ (baseline success $\rightarrow$ pipeline failure) and $c=295$ (baseline failure $\rightarrow$ pipeline success), giving $p=1.10\times10^{-65}$. This confirms that the observed improvement is statistically significant and driven by recovered single-shot failures.

Convergence reliability was evaluated on the full set of 604 prompt--model cases (151 prompts across 4 backends), with observed eventual acceptance in all 604/604 cases. Using the exact Clopper--Pearson bound for the all-success case, the 95\% lower confidence bound is $p_L=(0.025)^{1/604}\approx 0.9939$ (99.39\%)~\cite{clopper1934confidenceLimits}. Thus, with 95\% confidence, convergence probability is at least 99.39\% for SysMBench-style prompts under the evaluated controller, validator, and backend configuration.

Equivalently, for failure probability $q=1-p$, the rule-of-three gives $q\leq 3/604\approx 0.00497$ (0.497\%)~\cite{hanley1983zeroNumerators}. At the 95\% level, this corresponds to an upper bound of approximately 0.5\% on non-convergence under the same scope.

These bounds are intentionally scoped and do not claim universal convergence for arbitrary natural-language inputs. However, because convergence is driven by deterministic validator diagnostics, accepted cases terminate immediately, and repairs are guided by structured feedback rather than prompt memorization, the observed reliability is consistent with a structural property of the validator-in-the-loop control mechanism on this benchmark distribution.

\subsection{Convergence Behavior}
Convergence is indexed by repair cycles: $k=0$ denotes the initial single-shot generation (no validator feedback), and $k\geq 1$ denotes validator-guided repair cycles; acceptance corresponds to zero-error output under the production validator.

Figure~\ref{fig:cumulative_convergence} first presents the pooled convergence trajectory across all prompt-level cases.

\begin{figure}[H]
\centering
\definecolor{pipelineColor}{RGB}{44,138,100}
\begin{tikzpicture}
\begin{axis}[
    width=\columnwidth,
    height=0.60\columnwidth,
    xmin=0,
    xmax=7,
    ymin=0,
    ymax=105,
    xlabel={Repair Cycles (k)},
    ylabel={Cumulative Acceptance (\%)},
    xtick={0,1,2,3,4,5,6,7},
    ymajorgrids=true,
    xmajorgrids=false,
    grid style={dashed,gray!30},
    axis lines*=left,
    tick label style={font=\footnotesize},
    label style={font=\footnotesize},
]
\addplot+[pipelineColor, very thick, mark=*, mark size=1.8pt] coordinates {
    (0,51.16)
    (1,84.44)
    (2,94.37)
    (3,98.18)
    (4,99.67)
    (5,99.67)
    (6,99.83)
    (7,100.00)
};
\addplot+[black!45, dashed, thin] coordinates {(0,100) (7,100)};
\end{axis}
\end{tikzpicture}

\caption{Cumulative production-validation acceptance versus repair cycles ($k$). $k=0$ denotes initial single-shot generation; $k\geq 1$ denotes validator-guided repair cycles. Acceptance corresponds to zero-error output under the production validator.}
\label{fig:cumulative_convergence}
\end{figure}

The pooled curve shows a large first-step jump from $k=0$ to $k=1$, followed by rapid compression by $k=2$ and then a short tail. Table~\ref{tab:convergence_behavior} provides the exact counts and percentages underlying Figure~\ref{fig:cumulative_convergence}: acceptance increases from 51.16\% at $k=0$ to 84.44\% at $k=1$, reaches 94.37\% by $k=2$, and reaches 99.67\% by $k=4$. Only two cases remain in the grouped $k=5$--$8$ tail (0.33\%), where cumulative acceptance reaches 100.00\%. Consistent with this front-loaded pattern, iterations-to-acceptance are summarized by mean 1.727, median 1, IQR 1--2, and maximum 8.

\begin{table}[H]
\centering
\caption{Distribution of repair cycles to first production-validation acceptance (pooled across 604 prompt-level cases).}
\label{tab:convergence_behavior}
\begin{tabular}{lrrr}
\toprule
Repair cycles ($k$) & Cases & Share & Cumulative \\
\midrule
0 & 309 & 51.16\% & 51.16\% \\
1 & 201 & 33.28\% & 84.44\% \\
2 & 60 & 9.93\% & 94.37\% \\
3 & 23 & 3.81\% & 98.18\% \\
4 & 9 & 1.49\% & 99.67\% \\
5--8 & 2 & 0.33\% & 100.00\% \\
\bottomrule
\end{tabular}

\end{table}


\noindent\textbf{Rate Characterization.} To quantify the observed convergence shape, we report an empirical contraction analysis of residual failure mass. Under contraction-style iteration analysis, let residual failure mass be $R_k = 1 - A_k$. The observed residual sequence is $R_0=0.4884$, $R_1=0.1556$, $R_2=0.0563$, $R_3=0.0182$, and $R_4=0.0033$. Using early cycles, the empirical contraction ratios are $\rho_0 \approx 0.1556/0.4884 \approx 0.32$, $\rho_1 \approx 0.0563/0.1556 \approx 0.36$, and $\rho_2 \approx 0.0182/0.0563 \approx 0.32$. These values cluster around an average early-cycle contraction factor of approximately 0.33 (computed as the arithmetic mean of $\rho_0$--$\rho_2$). Tail transitions are excluded from contraction-rate estimation because ratios become unstable when residual mass is near the finite-sample resolution. These early-cycle ratios cluster in a narrow band, suggesting an approximately multiplicative (``contraction-like'') reduction pattern in the initial repair regime. In practical terms, early cycles remove roughly two-thirds of the remaining failures per cycle (in the observed early regime).

Complementing the contraction estimate, time-to-threshold metrics quantify convergence speed. The observed thresholds are $T_{90}=2$, $T_{95}=3$, and $T_{99}=4$. These values characterize how rapidly validator-guided refinement approaches the acceptance set in repair-cycle space.

To assess whether this pooled behavior is shared across backends, Figure~\ref{fig:cumulative_convergence_by_model} overlays per-model cumulative acceptance trajectories.

\begin{figure}[H]
\centering
\definecolor{sonnetColor}{RGB}{31,119,180}
\definecolor{openaiColor}{RGB}{255,127,14}
\definecolor{deepseekColor}{RGB}{44,160,44}
\definecolor{mistralColor}{RGB}{214,39,40}
\begin{tikzpicture}
\begin{axis}[
    width=\columnwidth,
    height=0.64\columnwidth,
    xmin=0,
    xmax=7,
    ymin=0,
    ymax=102,
    xlabel={Repair Cycles (k)},
    ylabel={Cumulative Acceptance (\%)},
    xtick={0,1,2,3,4,5,6,7},
    ymajorgrids=true,
    xmajorgrids=false,
    grid style={dashed,gray!30},
    axis lines*=left,
    tick label style={font=\footnotesize},
    label style={font=\footnotesize},
    legend style={
        draw=none,
        font=\footnotesize,
        at={(0.5,1.02)},
        anchor=south,
        legend columns=2
    },
]
\addplot+[sonnetColor, very thick, mark=*] coordinates {
    (0,82.78) (1,97.35) (2,99.34) (3,99.34) (4,99.34) (5,99.34) (6,100.00) (7,100.00)
};
\addplot+[openaiColor, very thick, mark=square*] coordinates {
    (0,41.72) (1,88.08) (2,98.01) (3,99.34) (4,100.00) (5,100.00) (6,100.00) (7,100.00)
};
\addplot+[deepseekColor, very thick, mark=triangle*] coordinates {
    (0,41.06) (1,76.82) (2,89.40) (3,97.35) (4,99.34) (5,99.34) (6,99.34) (7,100.00)
};
\addplot+[mistralColor, very thick, mark=diamond*] coordinates {
    (0,39.07) (1,75.50) (2,90.73) (3,96.69) (4,100.00) (5,100.00) (6,100.00) (7,100.00)
};
\legend{Anthropic Sonnet 4.6,OpenAI Codex 5.2,DeepSeek Reasoner,Mistral Large}
\end{axis}
\end{tikzpicture}

\caption{Per-model cumulative production-validation acceptance versus repair cycles ($k$).}
\label{fig:cumulative_convergence_by_model}
\end{figure}

Figure~\ref{fig:cumulative_convergence_by_model} shows that single-shot starting points differ across models, but the trajectories compress rapidly once validator-guided repair begins. Despite different initial acceptance levels, all curves move quickly toward full acceptance within a small number of repair cycles. This pattern supports the model-agnostic control-signal interpretation: deterministic validator diagnostics drive the dominant convergence dynamics across backends.

\subsection{Error Family Distribution}
First-iteration failures were concentrated in a small number of recurring validator error families. Pooled across models, first-iteration diagnostics contained 2,209 errors, dominated by \texttt{parsing-error} and \texttt{reference-error}, which together account for 89.09\% of first-iteration errors. This concentration indicates that most initial failures are structural or reference-resolution issues rather than diffuse long-tail failure modes.

\begin{table}[H]
\centering
\caption{Top first-iteration validator error families (pooled across all models).}
\label{tab:error_family_distribution}
\begin{tabular}{lrr}
\toprule
Error family & Count & Share \\
\midrule
parsing-error & 1144 & 51.79\% \\
reference-error & 824 & 37.30\% \\
port-definition-owned-usages-not-composite & 82 & 3.71\% \\
feature-chaining-feature-not-one & 28 & 1.27\% \\
type-error & 26 & 1.18\% \\
\bottomrule
\end{tabular}

\end{table}


\subsection{Secondary Diagnostics}
Secondary diagnostics are consistent with the primary outcome but are not the focus of this paper. Recovery among single-shot failures was complete (295/295). Prompt-level difficulty showed a long tail, with a few high-burden cases requiring substantially more repair than typical prompts. Runtime and token usage also varied by model despite identical pipeline pass outcomes.

\subsection{Open-Source Dataset Release}
This campaign releases an open-source dataset of 604 generated SysMLv2 outputs (151 prompts $\times$ 4 model backends), together with run-level records, iteration histories, and validator diagnostics in the project repository~\cite{sysmbenchCompilerLoopRepo2026}. This provides a reproducible, inspectable benchmark-scale corpus for follow-on syntactic-reliability studies.


\section{Discussion}
We discuss why deterministic compiler diagnostics provide an effective supervision signal for new, sparsely represented languages like SysMLv2, and the resulting trade-offs in runtime and number of iterations.

\section{Conclusion}
We present a reproducible pathway for reliable natural-language to SysMLv2 translation by embedding deterministic compiler feedback into LLM generation, thereby guaranteeing syntactically valid (compilable) output by construction.

\section*{Acknowledgments}

\bibliographystyle{asmeconf}
\bibliography{references}

\end{document}

%%%%%%%%%%%%%%%%%%%%%%%%%%%%%%%%%%%%%%%%%%%%%%%%%%%%%%%%%%%

\section{Referring to Citations, Figures, and Equations}

Citations are automatically numbered \cite{ning2002}. They should be inserted in the text using a \verb|\cite{ref}| command~\cite{gibson2008,stevens1999}. The citations will be automatically sorted and compressed if they are given in a set \cite{stevens1999,ning2002,gibson2008,wions2005,smith2002,watson1982}. 
A specific reference may be named with an abbreviation, as in Ref.~\cite{watson1982}.
See the \texttt{asmeconf-sample.bib} file and Sect.~\ref{sec:references} for examples of entering references.

For ASME conference papers, the labels Equation and Figure should be abbreviated when they do not start a sentence, as in  Eq.~\eqref{eqn:dw} and Fig.~\ref{fig:1}. Figure~\ref{fig:1} is spelled out when it starts a sentence. Equation~\eqref{eqn:dw} is spelled out when it starts a sentence. 

Equations are typeset in the usual way and will be automatically numbered.  The class file loads the \texttt{amsmath} and \texttt{mathtools} packages. Further, the \texttt{newtxmath} package used for the math fonts includes many additional math features (see Sect.~\ref{sec:moremath}).
\begin{equation}\label{eqn:fourier}
\vec{q} = -k\nabla T
\end{equation}

ASME prefers SI units. (U.S.\ style units may follow in parentheses.) Be sure to put all symbols into the nomenclature list, including their units.


%%%%%%%%%%%%% begin figure %%%%%%%%%%%%%%%%%

%% captions go below figures

\begin{figure}
\centering\includegraphics[width=\linewidth, alt = {Linearization error in radiant flux}]{sample-figure-1.pdf}
\caption{Caption with math, eqn.~\eqref{eqn:fourier}, $\Delta T/T_m$ vs.\ $\Delta T/T_1$~\cite{Lienhard2019}}\label{fig:1}
\end{figure}
 
%%%%%%%%%%%%% end figure %%%%%%%%%%%%%%%%%%%


%%%%%%%%%%%%%%%%%%%%%%%%%%%%%%%%%%%%%%%%%%%%%%%%%%%%%%%%%%%

%% Use title case for subsections and subsubsections (first letter of words capitalized)

\section{Section Headings and Captions}

ASME requires that section headings and captions be set in an uppercase, sans serif font.  The class will do this automatically.  You can place \verb|\cite{..}|, \verb|\ref{..}|, \verb|\label{..}|, and mathematics into headings and captions directly, as you would in the main text. Do not enclose them braces, e.g.\ \verb|{\cite{..}}|, which will cause errors. You can place \verb|\footnote{..}| into headings, but not into captions.\footnote{See \texttt{tex-stackexchange} for various approaches to footnotes in captions, if they seem necessary. For footnotes in tables, use the \texttt{tablefootnote} package.}\footnote{Sequential footnotes are automatically separated by a comma.}

Text in section headings and captions will not be capitalized if enclosed in a \verb|\NoCaseChange{..}| command.

Sections may either be numbered or left unnumbered.

Simple mathematical expressions can be used in either captions or section headings. For a section heading that includes more complicated math (and macros), you may use the optional argument of \verb|\section[..]{..}| to create a pdf bookmark without losing characters or producing warnings or errors. See the \texttt{asmeconf-template.tex} source file for examples of this procedure. These bookmarks should usually be text expressions, although some math is supported.  

To eliminate boldface type in caption text and math, use the class option \texttt{[unboldcaption]}.  To prevent sans-serif math, put \verb|\NoCaseChange{\mathversion{normal}}| in the caption.

\subsection{Subsection and Sub-subsection Headings}

Subsections and sub-subsection headings should be entered in title case, with the first letter of primary words capitalized. Sub-subsections (i.e., paragraphs) are never numbered.


%%%%%%%%%%%%%%% begin simple table %%%%%%%%%%%%%%%%%%%%%%%%%% 

%% Captions go above tables
%%
%% Note that placement of figures and tables can managed with the [!tbhp] options. See: https://latexref.xyz/Floats.html

\begin{table}[t]
\caption[Table]{A simple table}\label{tab:1}
\centering{%
\tagpdfsetup{table/header-rows={1}}%
\begin{tabular}{llr}
\toprule
Experiment & $u$ [m/s] & $T$ [\textdegree C] \\
\midrule
Run 11 & 12.5 & 103.4 \\
Run 12 & 24   & 68.3 \\
\bottomrule
\end{tabular}
}
\end{table}

%%%%%%%%%%%%%%%% end table  %%%%%%%%%%%%%%%%%%%%%%%%%%%%%%%%%%%% 

%%%%%%%%%%%%%%% begin more complicated table %%%%%%%%%%%%%%%%%%%%%%%%%%%%%%%%%%%%

% note column set as a 3 cm wide paragraph with 1 em hanging indentation. See array package documentation.
% note numbers centered on "." (with d{3.3}) and on "," (with ,{3.3}).  See dcolumn package documentation.
\begin{table}[t]
\caption{Table with more complicated columns}\label{tab:2}%
\centering{%
\tagpdfsetup{table/header-rows={1}}%
\begin{tabular}{!{\hspace*{0.5cm}} >{\raggedright\hangindent=1em} p{3cm} d{3.3} @{\hspace*{1cm}} ,{3.3} !{\hspace*{0.5cm}}}
\toprule
Experiment & \multicolumn{1}{c@{\hspace*{1cm}}}{$u$ [m/s]} & \multicolumn{1}{c!{\hspace*{0.5cm}}}{$T$ [\textdegree C]} \\
\midrule
The first test we ran this morning   & 124.3     &   68,3   \\
The second test we ran this morning  &  82.50    &  103,46  \\
Our competitor's test                &  72.321   &  141,384 \\
\bottomrule
\end{tabular}
}
\end{table}

%%%%%%%%%%%%%%%% end table  %%%%%%%%%%%%%%%%%%%%%%%%%%%%%%%%%%%% 

%%%%%%%%%%%%%%% begin two column table %%%%%%%%%%%%%%%%%% 
\begin{table*}
\caption{A table spanning two columns}\label{tab:3}%
\centering{%
\tagpdfsetup{table/header-rows={1}}%
\begin{tabular*}{0.8\textwidth}{@{\hspace*{1.5em}}@{\extracolsep{\fill}}ccc!{\hspace*{3.em}}ccc@{\hspace*{1.5em}}}
\toprule
\multicolumn{1}{@{\hspace*{1.5em}}c}{$x$\rule{0pt}{11pt}} &
\multicolumn{1}{c}{$\textrm{erf}(x)$} &
\multicolumn{1}{c!{\hspace*{3.em}}}{$\textrm{erfc}(x)$} &
\multicolumn{1}{c}{$x$} &
\multicolumn{1}{c}{$\textrm{erf}(x)$} &
\multicolumn{1}{c@{\hspace*{1.5em}}}{$\textrm{erfc}(x)$} \\ \midrule
0.00 & 0.00000 & 1.00000 & 1.10 & 0.88021 & 0.11980\rule{0pt}{11pt} \\
0.05 & 0.05637 & 0.94363 & 1.20 & 0.91031 & 0.08969 \\
0.10 & 0.11246 & 0.88754 & 1.30 & 0.93401 & 0.06599 \\
0.15 & 0.16800 & 0.83200 & 1.40 & 0.95229 & 0.04771 \\
0.20 & 0.22270 & 0.77730 & 1.50 & 0.96611 & 0.03389 \\
0.30 & 0.32863 & 0.67137 & 1.60 & 0.97635 & 0.02365 \\
0.40 & 0.42839 & 0.57161 & 1.70 & 0.98379 & 0.01621 \\
0.50 & 0.52050 & 0.47950 & 1.80 & 0.98909 & 0.01091 \\
0.60 & 0.60386 & 0.39614 & 1.82\makebox[0pt][l]{14} & 0.99000 & 0.01000 \\
0.70 & 0.67780 & 0.32220 & 1.90 & 0.99279 & 0.00721 \\
0.80 & 0.74210 & 0.25790 & 2.00 & 0.99532 & 0.00468 \\
0.90 & 0.79691 & 0.20309 & 2.50 & 0.99959 & 0.00041 \\
1.00 & 0.84270 & 0.15730 & 3.00 & 0.99998 & 0.00002 \\[2pt]
\bottomrule\end{tabular*}
}
\end{table*}

%%%%%%%%%%%%%%%%% end two column table  %%%%%%%%%%%%%%%%%%%%%%%%%%%%%%% 

%%%%%%%%%%%%%%%%%%%%%%%%%%%%%%%%%%%%%%%
\section{Tables and Figures}

Table \ref{tab:1} is an example of a simple table. Table captions should be placed above tables.
The class loads the \texttt{booktabs} package (used for horizontal rules in Tables \ref{tab:1} and \ref{tab:2}), and the \texttt{array} and \texttt{dcolumn} packages which provide extended capabilities for columns in the \texttt{tabular} environment (see Table \ref{tab:2}).  Table \ref{tab:3} is an example of a table that spans two columns. Two column tables (and figures) will always float to the top of a later page.

Figure captions go below figures. Figure~\ref{fig:2} is an example of a figure that spans two columns and includes subfigures. The text in figures (and tables) should be no smaller than 6~point type. Images in figures are handled by the standard \texttt{graphicx} package.

Landscape figures and tables may be produced at full-page size by putting \verb|\usepackage[figuresright]{rotating}| in your \texttt{.tex} file's preamble and using the \texttt{sidewaystable*} and \texttt{sidewaysfigure*} environments~\cite{fairbairns}.


%%%%%%%%%%%%%%%%%%%%%%%%%%%%%%%%%%%%%%%%%%%%%%%%%%%%%%%%%%%%%%%%%%%%%%

\section{Reference Formatting with \NoCaseChange{\texttt{asmeconf.bst}}\footnote{To prevent capitalization of text in a section heading or caption, such as an SI unit, enclose it in a \texttt{\textbackslash NoCaseChange} command. As of the July 2022 release of \LaTeX, commands used in a heading or caption may be protected globally by putting this in the preamble: \texttt{\textbackslash AddToNoCaseChangeList\{\textbackslash MyCommand\}}}}\label{sec:references}

The {\upshape\texttt{asmeconf.bst}} \BibTeX\   style follows the reference styles shown on ASME's conference web site in  2025.\footnote{\texttt{asmeconf.bst} is intended as a replacement for the old \texttt{asmems4.bst}, which does not follow ASME's current reference formats or support DOI and URL.}
Examples for these and many other cases are given in the \texttt{asmeconf-sample.bib} file, which is part of this distribution. Citations and references are managed by the standard \texttt{natbib} package.  Nevertheless, a few comments are necessary. 

%% sub-subsections should *not* be numbered according to ASME's style

\subsubsection*{DOI, URL, and eprint} Include DOI numbers when they are available.  URL's may alternatively be given. ASME requests that URLs point to a document's abstract.

Basic support for \texttt{eprint} numbers is also included, generating a url at the end of the citation. The \texttt{archive} type may be specified using the macros \texttt{arxiv, google\-books, hdl, jstore, oclc}, or \texttt{pubmed} (e.g., \texttt{archive=hdl},  \emph{without} braces). Both \texttt{eprint} and \texttt{archive} fields \emph{must} be given. Other root urls may be invoked using \verb|archive = {https://another.url.org/}|.

\subsubsection*{Online Sources} A bibliography entry \verb|@online{..| is included for citation of online sources, such as web pages. A \texttt{url} or \texttt{eprint} with \texttt{archive} must be included. See the examples of use in the \texttt{asmeconf-sample.bib} file. 

\subsubsection*{Date Accessed} The \verb|urldate={..}| field may be used to provide the date on which a given url was accessed. By default, the text printed will be \texttt{Accessed `date',}. The word ``Accessed'' may be changed using the \verb|urltype={..}| field.

\subsubsection*{Conference Location and Date} To specify the city and date of a conference, you can use \verb|venue={..}| and \verb|eventdate={..}| with the entries \verb|@inproceeedings{..| and \verb|@proceeedings{..|

\subsubsection*{Capitalization of Titles} ASME's bibliography style requires that document titles be in title case. The first letters of principal words are capitalized. Do this in the \texttt{.bib} file.



%%%%%%%%%%%%%%%%%  begin two column figure  %%%%%%%%%%%%%%%%%%%%%%%%%%%

\begin{figure*}
\begin{subfigure}[b]{\columnwidth}% subfigure is basically the same as minipage
\centering{
  \includegraphics[width=0.9\linewidth, alt={Nusselt number data for isothermal wall}]{sample-figure-2a.pdf}%
}
\subcaption{Uniform temperature wall\label{fig:isothermal}}
\end{subfigure}%
\hspace*{\columnsep}% with this space added, puts each figure at column center
%%%%%%%%%%%%% no line break between these two subfigures
\begin{subfigure}[b]{\columnwidth}
\centering{%
\includegraphics[width=0.927\linewidth, alt={Nusselt number data for uniform heat flux wall}]{sample-figure-2b.pdf}%
}%
\subcaption{Uniform heat flux wall with unheated starting length\label{fig:uniform-flux}}
\end{subfigure}
\caption{A figure with two subfigures~\cite{lienhard2020}\label{fig:2}}
\end{figure*}

%%%%%%%%%%%%%%%%%%%  end two column figure  %%%%%%%%%%%%%%%%%%%%%%%%%%



%%%%%%%%%%%%%%%  MORE ON MATH   %%%%%%%%%%%%%%%%%%%%%%%%%%%%%%%%%%%%%%%%%%%%%%%%%%%%%%%%%%%%%%%%%%%%%%

%% Here is an example of managing complicated math in a section or subsection heading: 
%%    the optional argument to \section will provide the pdf bookmark
%%    without losing most characters or producing warnings/errors.
%%
%% In this heading, letter u is forced to be upright with \mathrm{u}
%%
\section[More on math: u*omega=0]{More on math: $\vec{\mathrm{u}}\cdot\vec{\omega}=0$}\label{sec:moremath}

In most cases, the need for a wide equation can be eliminated by using one of the multiline equation environments defined by 
\texttt{amsmath}, such as \texttt{align}, \texttt{split}, or \texttt{multline}~\cite{amsmath}. The following equation is set with the 
\texttt{multline} environment:
\begin{multline}\label{eqn:energy}
\frac{\partial}{\partial t}\left[\rho\bigl(e + \lvert\vec{u}\rvert^2\big/2\bigr)\right]  + \nabla\cdot\left[\rho\bigl(h + \lvert\vec{u}\rvert^2\big/2 \bigr)\vec{u}\right] \\
 ={}-\nabla \cdot \vec{q} +  \rho \vec{u}\cdot\vec{g}+ \frac{\partial}{\partial x_j}\bigl(d_{ji}u_i\bigr) + \dot{Q}_v
\end{multline}
An example using \texttt{align} appears in Appendix~\ref{appendix:a}.

An experimental package for setting equations that span two columns, \texttt{asmewide.sty}, can be loaded as well, but that code may require hand-fitting around figures, tables, and page breaks. See the examples in~\cite{lienhard2022}. An alternative solution may be to set large equations into two-column-wide tables or figures. 

Math italics are used for Roman and Greek letters by default.  If you want an upright letter in math, you can use the relevant math alphabet, e.g., \verb|\mathrm, \mathbf, \mathsf|:
\begin{equation}\label{eqn:dw}
\vec{F} = m \vec{a} \quad\textrm{or}\quad \vec{\mathrm{F}} = m \vec{\mathrm{a}} \quad\textrm{or}\quad \mathbf{F} = m \mathbf{a} \quad\textrm{or}\quad \vec{\mathsf{F}} = m \vec{\mathsf{a}}
\end{equation}

\subsection{The \texttt{newtxmath} and \texttt{unicode-math} Packages~\cite{sharpe1,robertson2023}} The \texttt{newtxmath} package, loaded by default with \pdfTeX, includes many options for mathematics, most of which can be called as options to \verb|\documentclass|. For example, the \texttt{upint} option selects upright integral signs (rather than slanted integral signs):
\begin{quote}
\verb|\documentclass[upint]{asmeconf}|. 
\end{quote}  
The option \verb|subscriptcorrection| improves the spacing of math subscripts. Math options are discussed further in the \texttt{asmeconf-template.tex} file.  The \texttt{newtxmath} package is also loaded with the \texttt{[nofontspec]} option.

To get additional symbols in bold math with \pdfTeX, use \verb|\bm{..}| from the \texttt{bm} package, which is loaded by the class. 

If using \LuaLaTeX, the math features of \texttt{unicode-math} are available.  These include commands to select a boldface, upright symbol, \verb|\symbfup{..}| or \verb|\mathbfup{..}|, to select boldface fraktur 
symbol, \verb|\symbffrak{..}| or \verb|\mathbffrak{..}|, and so on.  See the documentation of \texttt{unicode-math} for details~\cite{robertson2023}.

The \texttt{[upint]} option also works under \LuaLaTeX.

For longer passages of bold math, you can use \verb|{\mathversion{bold}..}| with either \pdfTeX\  or \LuaLaTeX: 
\verb|{\mathversion{bold} $A\otimes\mathfrak{F}$ }| gives {\mathversion{bold} $A\otimes\mathfrak{F}$}. Note that the math version must be changed \emph{before} starting math mode.


\subsection{Sans-Serif Greek Letters}
The class file also provides upright sans-serif Greek letters with \verb|\sfalpha| and similar expressions (e.g., $\sfalpha, \sfbeta, \sfgamma, \sfdelta$ \ldots ), in case they are needed. 

Under \pdfTeX\  boldface, upright, sans-serif Greek letters can be obtained with \verb|\bm{\sfalpha}|, etc.  The \texttt{newtx} package also includes options that affect whether Greek letters are upright or slanted (see that package's documentation for details).

Under \LuaLaTeX, boldface, sans-serif, upright Greek can be obtained with \verb|\sfbfalpha|, etc., or by using \verb|\symbfsfup{\alpha}|.  In the first case, the glyphs are drawn from the Lete Sans Math font, while in the second case, the glyphs are from the TeX Gyre Termes Math font.   Why are two different fonts being used? Unicode does not include medium-weight, upright, sans-serif Greek as a component of a serif font (like Termes), although it does include both upright and italic boldface Greek. However, in a sans-serif font (like Lete), Greek letters are sans-serif by default, including medium weight. (You can also invoke \verb|\symbfsfit{\alpha}| for  boldface, sans-serif, slanted type.) 

\subsubsection*{Sans-serif Math Versions} Two additional math versions, \texttt{sansbold} and \texttt{sans} are available in \texttt{asmeconf}.  The former is used by default in captions and section headings.  The latter is available in case you find a use for it. In \LuaLaTeX, both versions use the Lete Sans Math fonts. In \pdfTeX, the glyphs are from the \texttt{newtxsf} package.
  
  
\subsection{Controlling Calligraphic, Script, Fraktur, or BB Fonts}
With \pdfTeX, the \texttt{[mathalpha]} package may be loaded in the preamble~\cite{sharpe2}.\footnote{As of v1.41, the \texttt{[mathalfa]} class option has been dropped.} This package supports variety of font for calligraphic, fraktur, script, and blackboard bold fonts. For example,
\begin{center}
\verb|\usepackage[cal=euler,frak=boondox]{mathalpha}| 
\end{center}
selects the Euler font for \verb|\mathcal| and the Boondox font for \verb|\mathfrak|. Refer to the \texttt{mathalpha} documentation for details~\cite{sharpe2}.  The \texttt{[nofontspec]} option also supports \texttt{mathalpha}.

Under \LuaLaTeX, the \texttt{unicode-math} \texttt{range} function can be used to select such fonts~\cite{robertson2023}. For example, the following code in the preamble would select the Euler Math font for calligraphic, script, fraktur, and blackboard bold fonts:
\begin{quote}\raggedright
\verb|\setmathfont{Euler-Math}[|
\verb| range={cal,scr,frak,bb},|
\verb| Extension=.otf,Scale=MatchUppercase]|
\end{quote}


%%%%%%%%%%%%%%%  ADDITIONAL PACKAGE OPTIONS  %%%%%%%%%%%%%%%%%%%%%%%%%%%%%%%%%%%%%%%%%%%%%%%%%%%%%%

\section{Additional Options for \NoCaseChange{\texttt{asmeconf.cls}}}
The class accepts a number of options in addition to those already described. These options are discussed next.

\subsection{Colored Hyperlinks}
ASME requires that all text be \textbf{in black} when the paper is submitted for publication.  For other uses, authors may
obtain colored hyperlinks with the [\texttt{colorlinks}] option.

\subsection{Final Column Balancing} The option \texttt{[balance]} invokes the the \texttt{flushend} package~\cite{tolusis}.
This package will attempt to give equal height to the two columns on the last page. The performance of this package is sometimes inconsistent (with odd page layout or, very rarely, errors), so use this option with caution.

\subsection{Grid-Style Author Block} The option \texttt{[grid]} invokes ASME's grid-style arrangement of author names. In the \verb|\SetAuthors{..}| command, individual author's names are recognized by the commas that separate them. (To include a comma \emph{in} a name, enclose the name in braces.) Line breaks (\verb|\\|) may be inserted into the address of \verb|\SetAffiliation{n}{address}| as needed. 

Note that ASME interprets the author order in the grid style by reading names from left-to-right in the top row, then left-to-right in each subsequent row.

\subsection{Line Numbers} The option \texttt{[lineno]} invokes the the \texttt{lineno} package~\cite{bottcher}. This option will produce line numbers in the margins. You must run \LaTeX\ \emph{twice} for proper placement of the numbers. Tables, captions, and footnotes will not be numbered.  Line numbers can be helpful for review and editing, but should not be used in your final manuscript. See the documentation of the \texttt{lineno} package for further commands to control line numbering. 

The \texttt{lineno} package is not compatible with the \texttt{flushend} package that makes final short columns the same height. Balancing is automatically disabled when this option is called. 

\subsection{Changing the Copyright Footer} The option \texttt{[nofoot]} will omit the ASME copyright from the page footer. The option \texttt{[govt]} will produce a copyright notice for authors who are employees of the U.\ S.\ Government.  
The option \texttt{[contractor]} will produce a copyright
notice for authors who are employed by a U.\ S.\ Government contractor.
The option \texttt{[somegovt]} gives a copyright notice for the case when only some authors are employees of the U.\ S.\ Government.

The footers are generated with the \texttt{fancyhdr} package~\cite{oostrum} and can be changed using the commands of that package. Only the default arrangement matches ASME's style, however.

In addition, the conference header on the title page can be omitted using the option \texttt{[nohead]}.

\subsection{Archivability:~PDF/A} In June 2022, the \LaTeX{}3 team added support for PDF/A to the \LaTeX\ kernel through the command \verb|\DocumentMetadata{..}|. This approach works with \emph{both} \pdfTeX\  and \LuaLaTeX. Note that accessible  conformance~(\texttt{a} or \texttt{UA-2} level, a.k.a.\ ``well-tagged PDF'') is still under development by the  \LaTeX3 team.

As of \texttt{asmeconf} v1.41, the legacy options \texttt{[pdf-a]}, \texttt{[pdfapart=]}, and \texttt{[pdfaconformance=]} have been dropped.

\subsection{Typewriter Font Options} This font is the sans-serif \texttt{inconsolata}. By default, the word spacing is variable, but option \texttt{[mono]} switches to monospacing. A slashed zero is the default; option \texttt{[var0]} removes the slash. Option \texttt{[hyphenate]} enables hyphenation. (The hyphenation option is not available under \LuaLaTeX\  with \texttt{fontspec}.)

\subsection{Support for Other Languages}  This package can be adapted to incorporate (or entirely use) languages other than English. See Appendix \ref{appendix:c} for details.


%%%%%%%%%%%%%%%  Nomenclature Environment  %%%%%%%%%%%%%%%%%%%%%%%%%%%%%%%%%%%%%%%%%%%%%%%%%%%%%%

\section{Nomenclature Environment}
A nomenclature environment is included, as illustrated just after the abstract.  Each item in the nomenclature list is entered as \verb|\entry{symbol}{meaning}|.  Optional subheadings can be included as well: \verb|\EntryHeading{Roman letters}|.  The environment includes an optional argument for changing the space between symbols and definitions, 
\verb|\begin{nomenclature}[Xcm]|, where X is a number and cm can be replaced by any LaTeX dimensional unit: pt, in, ex, em, pc, etc. The default value is~2~em.

The title of the nomenclature can be also changed, e.g.\ \verb|\renewcommand*{\nomname}{List of Symbols}|



%%%%% Conclusions %%%%%%%%%%%%%%%%%%%%%%%%%%%%%%%

\section{Conclusion}
Provide a brief conclusion (3 to 4 lines).


%%%%% Acknowledgments %%%%%%%%%%%%%%%%%%%%%%%%%%%

\section*{Acknowledgments}
Place any acknowledgments here.


%%%  REFERENCES  %%%%%%%%%%%%%%%%%%%%%%%%%%%%%%%%
%%
%% Put your references into your .bib file in the usual way. Run latex once, bibtex once, then latex twice.
%% The asmeconf.bst style allows @inproceedings and @proceedings to include: 
%%		venue = {Location of Conference}, 
%%		eventdate = {Month, days},

\nocite{*}%% <=== Delete this line unless you want to typeset the entire contents of your .bib file !!

\bibliographystyle{asmeconf}  %% .bst file following ASME conference format. Do not change.
\bibliography{asmeconf-sample}%% <=== change this to the name of your bib file


%%%  APPENDICES  %%%%%%%%%%%%%%%%%%%%%%%%%%%%%%%%
\appendix

%% Note that appendices will be "numbered" A, B, C, ... etc. Use \section, not \section*
%% Equations will be numbered sequentially following those in the paper. Do not reset the equation counter.

%% Here we use the optional argument of section to control the pdf bookmark and prevent errors.
\section[The Vector Product A\times B]{The vector product $\vec{A}\times\vec{B}$}\label{appendix:a}

This brief illustration of an appendix shows the numbering of the appendix and equations. Equations are numbered
consecutively, following those in the paper. Consider $\rho \neq \textrm{fn}(p)$:
\begin{align}
\frac{d\Gamma}{dt} &{}= \frac{d}{dt} \int_{\mathcal{C}} \mathbf{u} \cdot d\mathbf{r}\\
				   &{}= \int_{\mathcal{C}} \frac{D\mathbf{u}}{Dt} \cdot d\mathbf{r} + \underbrace{\int_{\mathcal{C}} \mathbf{u}\cdot d\biggl( \frac{d\mathbf{r}}{dt}\biggr)}_{=\, 0} \\[-2pt]
                   &{}= \iint_{\mathcal{S}} \nabla \times \frac{D\mathbf{u}}{Dt}  \cdot d\mathbf{A}\\
                   &{}= \iint_{\mathcal{S}}  \nabla p \times \nabla \left( \frac{1}{\rho}\right) \cdot d\mathbf{A}
\end{align}

%%%%%%%%%%%%%%%%%%%%%%%%%%%%%%%%%%%%%%%%%%%%%%%%%%%%%%%%%%%%%%%%%%%%%%
\section[Use with LuaLaTeX]{Use with \NoCaseChange{LuaLaTeX} }\label{appendix:b}

The \LuaLaTeX\  engine is useful with \texttt{asmeconf} in at least three situations:

\begin{description}

\item[Executing lua code directly in your \LaTeX\ file.] With lua code, complicated functions can be plotted or numerical integration can be executed. An example file in the distribution demonstrates this capability~\cite{lienhard2025}. In this situation, you can use the class option \texttt{[nofontspec]} to stay with the \texttt{newtx} fonts. (This setting does not support for non-Latin alphabets.)

\item[Using complex alphabets.] With \pdfTeX, \texttt{asmeconf} supports Latin alphabets, as well as Cyrillic, Greek, and Vietnamese. With \LuaLaTeX\  with the \texttt{fontspec} package you can use non-Latin fonts available on your computer if you call the \texttt{[loadscripts]} option.\footnote{For Latin scripts (including English) under \LuaLaTeX, you \emph{must} have these OpenType fonts (\texttt{.otf}) installed (all are in TeX Live and will be present if your installation is complete and up-to-date): TeX Gyre Termes X, TeX Gyre Termes Math, TeX Gyre Heros, Inconsolatazi4, LeteSansMath, STIX Two Math. For Greek and Russian, the Noto Serif, Sans, and Mono TrueType fonts (\texttt{.ttf}) are used, and other Noto fonts are used for south and east Asian scripts. The Noto fonts are \textbf{not} currently in TeX Live (go here: \hrefurl{https://github.com/googlefonts/noto-fonts}{github.com/googlefonts/noto-fonts}).}\footnote{The \texttt{[loadscripts]} option was added in v1.42 of \texttt{asmeconf}, in order to reduce compilation time when non-Latin scripts are not needed.}   For Japanese in particular, use the class option \texttt{[japanese]}. See the example file in this distribution for more details~\cite{lienhard2021}.

\item[Producing tagged PDF.] Under \LuaLaTeX{}  with \texttt{unicode-math} (which loads by default), \texttt{asmeconf} PDF files can meet the PDF 2.0/UA-2 accessibility standard using appropriate arguments to \verb|\DocumentMetadata{..}| provided that: a) subcaptions are not used; b) any figure files meet the standard; and c) language options are used cautiously (some languages are compatible with tagging, but others currently are not). As of the Nov.~2025 released of \LaTeX{}, accessible PDF can be produced with the following settings:
\begin{quote}
	\verb|\DocumentMetadata{ | \newline
	\verb| pdfstandard = { ua-2 , a-4f },| \newline
	\verb| tagging = on }|
\end{quote}
Tagged PDF files can render accurately as HTML files (see \hrefurl{https://ngpdf.com}{ngpdf.com}). Version 1.44 of \texttt{asmeconf} includes a CSS style sheet control web appearance, \texttt{asmeconf-style.css}. 

\end{description}

For clarity, \texttt{fontspec} and \texttt{unicode-math} are automatically loaded when \texttt{asmeconf} is run under \LuaLaTeX. Those packages require that the necessary fonts are available on your computer.  If you wish to use \LuaLaTeX\  without the features of \texttt{unicode-math} and \texttt{fontspec}, use the class option \texttt{[nofontspec]}. 

Over the long-term, \LuaLaTeX\  will become the preferred engine for using \LaTeX\ (see \href{https://www.latex-project.org/news/latex2e-news/ltnews40.pdf}{\emph{LaTeX\ News}, Issue~40}). Access to \LuaLaTeX\  is different in each \LaTeX\ platform. Check the documentation for your platform to load \LuaLaTeX.

%%%%%%%%%%%%%%%%%%%%%%%%%%%%%%%%%%%%%%%%%%%%%%%%%%%%%%%%%%%%%%%%%%%%%%
\section{Multilingual Support}\label{appendix:c}

ASME publishes in English, but the \texttt{babel} package is loaded for 
users who may wish to include other languages. For example, an author might wish to include an appendix that provides the 
abstract in another language.

When more than one language option is included in \verb|\documentclass[..]{asmeconf}|, English will be 
set as the document's main language. (To choose a different main language, set \texttt{[main=..]}).
If no language options are given, the package defaults to English.  As examples, a passage in German is 
shown in  \selectlanguage{german}\appendixname~\ref{app:pohlhausen}\selectlanguage{english},
%\selectlanguage{french}\appendixname~\ref{app:fourier}\selectlanguage{english}, 
followed by abstracts in other languages.

% NOTE: The following template demo line contains raw UTF-8 glyphs (àáâä...).
% Some build setups can choke on this and stop compilation entirely.
% Comment it out to ensure robust compilation.
% The input encoding can be utf-8, as for these glyphs:
% %% If you have trouble with the next line, your file may not be saved in utf-8 format. You can delete that line to resolve the issue.
% \typeout{If you have trouble with the next line, your file may not be saved in utf-8 format. You can delete that line to resolve the issue. Under LuaLaTeX, you can load the fontspec will support these characters if you have the relevant systems fonts installed}%
% àáâäæãåā  èéęëêēė  îïíīįì ôöòóœøōõ ûüùúū çćč ł ñń ßśš ÿ žźż.

Fonts similar to Times/Helvetica are used when Greek, Vietnamese, or selected cyrillic-alphabet languages are called as options under {\upshape\pdfTeX}. Using {\upshape\LuaLaTeX}, which loads the fontspec package, many additional scripts are available; see the supplemental notes for such usage~\cite{lienhard2021}. Possibilities include Arabic, Bengali, Chinese, Devanagari (e.g., for Hindi), Hangul (for Korean), Kana (for Japanese), and Tamil. \emph{These options require an up-to-date \LaTeX\ installation.}

The \texttt{asmeconf} class defines several switches that can be used to call languages only when certain class options have been called, as \verb|\if...\fi|: \verb|\ifScriptsLoaded|, \verb|\ifFontspecLoaded|, \verb|\ifpdftex|, and \verb|\ifJapaneseLoaded|.

The bibliography style, \texttt{asmeconf.bst}, is designed in English and aimed at \BibTeX.  

%%%%%%%%%%%%%%%%%%%%%%%%%%%%%%%%%%%%%%%%%%%%%%%%%%%%%%%%%%%%%%%%%%%%%%

\begin{selectlanguage}{german}%
\section{Wärmeaustausch und Reibungswiderstand (\NoCaseChange{von} E. Pohlhausen)}\label{app:pohlhausen}
In einer strömenden Flussigkeit sind Wärmeleitung und Wärmekonvektion Vorgänge, die mit der inneren Reibung (oder Impulsleitung) und mit der Impulskonvektion große Aehnlichkeit besitzen. Mathematisch findet dies seinen Ausdruck in dem gleichartigen Bau der Differentialgleichungen, die einerseits für die Temperatur und anderseits für den Geschwindigkeitsvektor in der Flüssigkeit bestehen. Man kann daraus auf eine Beziehung
zwischen dem Wärmeaustanch und dem Reibungswiderstand schließen, die eine strömende Flüssigkeit an einem festen Körper hervorrufen. Dies ist zuerst von Prandtl ausgesprochen und durchgeführt worden, und zwar für turbulente Vorgänge, unter der vereinfachenden Annahme von Wärmequellen und -senken im Innern der Flüssigkeit~\cite{pohlhausen1921}. 
\end{selectlanguage}%


%\begin{selectlanguage}{french}%
%\section{Discours Préliminaire de Fourier}\label{app:fourier}
%Les causes primordiales ne nous sont point con­nues; mais elles sont assujetties à des lois simples et constantes, que l'on peut découvrir par l'obser­vation, et dont l'étude est l'objet de la philosophie naturelle. 
%
%La chale ur pénètre, comme la gravité, toutes les substances de l'univers, ses rayons occupent toutes les parties de l'espace. Le but de notre ouvrage est d'exposer les lois mathématiques que suit cet élé­ment. Cette théorie formera désormais une des branches les plus importantes de la physique gé­nérale~\cite{fourier1822}. 
%\end{selectlanguage}%
 
%\begin{selectlanguage}{spanish}%
%    \begin{abstract*}
%    Este es el resumen del artículo. Escribimos en español. Se describen el problema, los métodos y los resultados. También se incluyen referencias.
%    \end{abstract*}
%\end{selectlanguage}% edited by Aarón Montoya-Moraga

\begin{selectlanguage}{vietnamese}
    \begin{abstract*}
    Đây là phần tóm tắt của bài báo khoa học. Chúng tôi viết bằng tiếng Việt. Vấn đề, các phương pháp và các kết quả được mô tả trong phần này. Tài liệu tham khảo cũng được bao gồm.
    \end{abstract*}
\end{selectlanguage}% Checked and edited by Nguyen Le and Thao Nguyen

%% If you have trouble with the following passages, delete them and remove the associated language option from \documentclass[..].
\typeout{If you have trouble with the language passages in non-Latin scripts, your file may not be saved in utf-8 format, or your LaTeX format may be old, or you may not have the assumed font installed. You can delete those lines to resolve the issue.}

%% Examples of abstracts in other languages. The first three are intended for pdflatex, not lualatex.
\ifpdftex
    \begin{selectlanguage}{greek}%
        \begin{abstract*}
        Αυτή είναι η περίληψη του άρθρου. Χρησιμοποιούμε την ελληνική γλώσσα. Περιγράφεται το πρόβλημα, οι μέθοδοι και τα αποτελέσματα. Περιλαμβάνονται επίσης αναφορές.
        \end{abstract*}
    \end{selectlanguage}% edited by George Barbastathis   
%    
    \begin{selectlanguage}{russian}
        \begin{abstract*}
        Это резюме статьи. Пишем по русски. Описаны проблема, методы и результаты. Библиография также включена.%
        \end{abstract*}
    \end{selectlanguage}% edited by Steven Gerasimoff
\fi
%
%   
\ifScriptsLoaded %%% These passages require using lualatex, having the mentioned fonts on your computer, and using the option [loadscripts]
%
	\IfFontExistsTF{NotoSans}{%
    	\IfFontExistsTF{NotoSerif}{%
    	%
        	\begin{selectlanguage}{greek}
                \begin{abstract*}
                Αυτή είναι η περίληψη του άρθρου. Χρησιμοποιούμε την ελληνική γλώσσα. Περιγράφεται το πρόβλημα, οι μέθοδοι και τα αποτελέσματα. Περιλαμβάνονται επίσης αναφορές.
                \end{abstract*}
            \end{selectlanguage}% edited by George Barbastathis
        %  
            \begin{selectlanguage}{russian}
                \begin{abstract*}
                Это резюме статьи. Пишем по русски. Описаны проблема, методы и результаты. Библиография также включена.
                \end{abstract*}%
            \end{selectlanguage}% edited by Steven Gerasimoff
        %
        }{\ClassWarning{\ClassName}{I can't find NotoSerif.ttf, which is needed for Greek and Russian language text. Please install that font.}}
    }{\ClassWarning{\ClassName}{I can't find NotoSans.ttf, which is needed for Greek and Russian language text. Please install that font.}}
%
	\IfFontExistsTF{NotoSansCJK-Regular.ttc}{%
    	\IfFontExistsTF{NotoSerifCJKkr-Regular.otf}{% The NotoSerifCJK fonts are in CTAN, but not TeX Live.
	    %
            \begin{selectlanguage}{korean}
                \begin{abstract*}
                이것은 한국어로 쓰인 논문의 초록입니다. 문제, 방법 및 결과가 설명되어 있습니다. 참조도 포함됩니다.
                \end{abstract*}
            \end{selectlanguage}% edited by Hyung Won Chung.
        %
        }{\ClassWarning{\ClassName}{I can't find NotoSerifCJKkr-Regular.otf, which is needed for Korean language text. Please install that font and its bold version. Check that you also have NotoSansCJK and NotoSansMonoCJKkr.}}
    %
    	\IfFontExistsTF{NotoSerifCJKsc-Regular.otf}{%
            \begin{selectlanguage}{chinese-simplified}
                \begin{abstract*}
                这是文章的摘要。我们用中文书写,描述了问题,方法和结果,还包括了参考文献。
                \end{abstract*}
            \end{selectlanguage}% edited by Zi Hao Foo
        %
        }{\ClassWarning{\ClassName}{I can't find NotoSerifCJKsc-Regular.otf, which is needed for Chinese language text. Please install that font and its bold version. Check that you also have NotoSansCJK and NotoSansMonoCJKsc.}}
    }{\ClassWarning{\ClassName}{I can't find NotoSansCJK-Regular.ttc, which is needed for Korean and simplified Chinese text. Please install that font and its bold version.}} 
\fi
%
%
%
% Use the class option [japanese], with lualatex and fontspec, to run this passage. (Note, Japanese has problems if pdf tagging is active.)
\ifJapaneseLoaded
    \IfFontExistsTF{NotoSerifCJKjp-Regular.otf}{%
    	\IfFontExistsTF{NotoSansCJK-Regular.ttc}{%
	    %
            \begin{selectlanguage}{japanese}
            \begin{abstract*}
            % 論文の要約です。日本語で記述します。問題、方法、および結果について説明します。また、参考文献も含めます。           
            この論文の日本語での要約は以下のとおりです。問題、方法、および結果が説明されています。参考資料も添付してあります。
            \end{abstract*}
            \end{selectlanguage}% Edited by Keiji Yano and Yoshiki Okamoto
        %
        }{\ClassWarning{\ClassName}{I can't find NotoSerifCJKjp, which is needed for Japanese language text. Please install that font.}}
    }{\ClassWarning{\ClassName}{I can't find NotoSansCJKjp, which is needed for Japanese language text. Please install that font.}}
\fi

%%%%%%%%%%%%%%%%%%%%%%%%%%%%%%%%%%%%%%%%%%%%%%%%%%%%%%%%%%%%%%%%%%%%%%%%%%%%%%%%%%%%%%%

\fi

\fi

\end{document}

 
